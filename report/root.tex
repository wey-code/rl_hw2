%%%%%%%%%%%%%%%%%%%%%%%%%%%%%%%%%%%%%%%%%%%%%%%%%%%%%%%%%%%%%%%%%%%%%%%%%%%%%%%%
%2345678901234567890123456789012345678901234567890123456789012345678901234567890
%        1         2         3         4         5         6         7         8

\documentclass[letterpaper, 10 pt, conference]{ieeeconf}  % Comment this line out if you need a4paper

%\documentclass[a4paper, 10pt, conference]{ieeeconf}      % Use this line for a4 paper

\IEEEoverridecommandlockouts                              % This command is only needed if 
                                                          % you want to use the \thanks command
\usepackage{subfigure}
\usepackage{cite}
\usepackage{amsmath,amssymb,amsfonts}
\usepackage{algorithmic}
\usepackage{graphicx}
\usepackage{textcomp}
\usepackage{xcolor}
%\usepackage[numbers,sort&compress]{natbib}
\usepackage{algorithm} %format of the algorithm 
\usepackage{multirow} %multirow for format of table 
\usepackage{amsmath} 
\usepackage{gensymb}
\usepackage{mathtools}
\usepackage{forloop}
\usepackage{soul}

\usepackage{algorithm}  

\usepackage[UTF8]{ctex} 



\soulregister\cite7
\setlength{\textfloatsep}{5pt}

\overrideIEEEmargins                                      % Needed to meet printer requirements.

%In case you encounter the following error:
%Error 1010 The PDF file may be corrupt (unable to open PDF file) OR
%Error 1000 An error occurred while parsing a contents stream. Unable to analyze the PDF file.
%This is a known problem with pdfLaTeX conversion filter. The file cannot be opened with acrobat reader
%Please use one of the alternatives below to circumvent this error by uncommenting one or the other
%\pdfobjcompresslevel=0
%\pdfminorversion=4

% See the \addtolength command later in the file to balance the column lengths
% on the last page of the document

% The following packages can be found on http:\\www.ctan.org
%\usepackage{graphics} % for pdf, bitmapped graphics files
%\usepackage{epsfig} % for postscript graphics files
%\usepackage{mathptmx} % assumes new font selection scheme installed
%\usepackage{times} % assumes new font selection scheme installed
%\usepackage{amsmath} % assumes amsmath package installed
%\usepackage{amssymb}  % assumes amsmath package installed



\title{\LARGE \bf
基于强化学习和注意力机制的车辆换道研究
}


\author{沈炼成,林镇阳,韩立君 }% <-this % stops a space


\UseRawInputEncoding
\begin{document}



\maketitle
\thispagestyle{empty}
\pagestyle{empty}


%%%%%%%%%%%%%%%%%%%%%%%%%%%%%%%%%%%%%%%%%%%%%%%%%%%%%%%%%%%%%%%%%%%%%%%%%%%%%%%%
\begin{abstract}
这里是华丽丽的摘要


\end{abstract}


%%%%%%%%%%%%%%%%%%%%%%%%%%%%%%%%%%%%%%%%%%%%%%%%%%%%%%%%%%%%%%%%%%%%%%%%%%%%%%%%
\section{简介}

\section{任务描述与分析}

整体任务为根据输入的车辆周围情况,借助深度强化学习算法得到高层的指令规划。再借助仿真器内部的运动规划器,将高层指令转化为具体的轨迹让底层控制器有效跟踪。

\subsection{仿真环境描述}
整体实验基于$highway\_env$开发,具有较强的灵活性。下面分别针对状态空间、动作空间等进行叙述。
\subsubsection{状态空间}
在环境中,状态空间可以选择底层的低维向量输入,也可以选择高维的图像输入和占用格作为输入。下面重点叙述使用低维向量输入和图片输入的基本情况。

当使用低维输入时,传入最近15辆车的坐标、速度、倾斜角度等信息表示出来,包括$x,y,vx,vy,cos_h,sin_h$。传入一个大小为$[15,7]$的数组。方便后面网络进行处理,其中第一行表示的是本车的未知




\section{强化学习算法}

\section{注意力机制}

\section{实验分析}



\addtolength{\textheight}{-1cm}   % This command serves to balance the column lengths
                                  % on the last page of the document manually. It shortens
                                  % the textheight of the last page by a suitable amount.
                                  % This command does not take effect until the next page
                                  % so it should come on the page before the last. Make
                                  % sure that you do not shorten the textheight too much.

%%%%%%%%%%%%%%%%%%%%%%%%%%%%%%%%%%%%%%%%%%%%%%%%%%%%%%%%%%%%%%%%%%%%%%%%%%%%%%%%



%%%%%%%%%%%%%%%%%%%%%%%%%%%%%%%%%%%%%%%%%%%%%%%%%%%%%%%%%%%%%%%%%%%%%%%%%%%%%%%%



%%%%%%%%%%%%%%%%%%%%%%%%%%%%%%%%%%%%%%%%%%%%%%%%%%%%%%%%%%%%%%%%%%%%%%%%%%%%%%%%
%\section*{APPENDIX}

%Appendixes should appear before the acknowledgment.

\section*{致谢}




%%%%%%%%%%%%%%%%%%%%%%%%%%%%%%%%%%%%%%%%%%%%%%%%%%%%%%%%%%%%%%%%%%%%%%%%%%%%%%%%

%References are important to the reader; therefore, each citation must be complete and correct. If at all possible, references should be commonly available publications.

\bibliographystyle{unsrt}
\bibliography{ref}%%我们的例子应该是\bibliography{cited}




\end{document}
